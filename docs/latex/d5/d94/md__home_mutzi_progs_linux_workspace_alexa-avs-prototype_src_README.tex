\href{https://developer.amazon.com/de/alexa-voice-service}{\tt } \section*{alexa-\/avs-\/prototype ( C++ )}

Alexa Voice Service (A\+VS) client in C++ for raspberry pi or any Linux Distribution -\/ This is a work in progress. Project is tested with Linux Gentoo(x86\+\_\+64) and Raspbian(\+A\+R\+M). \section*{What is A\+VS?}

Alexa Voice Service (A\+VS) is Amazon’s intelligent voice recognition and natural language understanding service that allows you as a developer to voice-\/enable any connected device that has a microphone and speaker. \section*{What you need?}


\begin{DoxyItemize}
\item A Linux Distribution with Pulse\+Audio Support.
\item Amazon Developer Account.
\item Follow the steps \href{https://github.com/alexa/alexa-avs-sample-app/wiki/Raspberry-Pi}{\tt here} to create a security profile.
\item From Security Profile, you need Product\+ID, Client\+ID, and Client\+Secret.
\item And a Wake\+Word\+Engine Client from \href{https://github.com/alexa/alexa-avs-sample-app/tree/master/samples/wakeWordAgent}{\tt here}.
\end{DoxyItemize}

\# Download it 
\begin{DoxyCode}
git clone https://github.com/blackmutzi/alexa-avs-prototype
cd ./alexa-avs-prototype
\end{DoxyCode}
 \# Build it 
\begin{DoxyCode}
qmake AVS-Prototype.pro
make -j2
\end{DoxyCode}
 \section*{A\+VS Configuration}

\paragraph*{First Step}

go to config directory and modify config.\+json file 
\begin{DoxyCode}
cd ./alexa-avs-prototype/config/
nano config.json 
\end{DoxyCode}
 and set only client\+\_\+id, client\+\_\+secret and product\+\_\+id. \paragraph*{Second Step}

run the A\+VS Prototype Client ... 
\begin{DoxyCode}
chmod +x ./AVS-Prototype
./AVS-Prototype
\end{DoxyCode}
 the A\+V\+S-\/\+Client show you a link, copy it. And login into the Amazon Security Profile Website. Then copy den code\+\_\+grant key and save it into the config.\+json. And Restart again A\+V\+S-\/\+Prototype. F\+I\+N\+I\+SH Configuration. A\+V\+S-\/\+Prototpye Client is now permanent R\+E\+A\+DY. \section*{D\+E\+P\+E\+N\+DS ( debian / raspbian packages )}


\begin{DoxyItemize}
\item libevent-\/dev ( version 2.\+0.\+5 )
\item libmp3lame-\/dev ( version 3.\+99.\+5 )
\item libcurl-\/dev ( version 7.\+54.\+0 )
\item libssl-\/dev ( version 1.\+0.\+2 -\/ A\+L\+PN h2 Protocol needed )
\item libboost-\/all-\/dev ( tested version with 1.\+55 , 1.\+63 )
\item libasound2-\/dev ( version 1.\+0.\+28 )
\item libnghttp2-\/dev ( version 1.\+18.\+1-\/1 , 1.\+22.\+0 , 1.\+24.\+0 or higher ) \section*{Known Bugs\+:}
\end{DoxyItemize}


\begin{DoxyItemize}
\item libboost-\/all-\/dev version 1.\+62 -\/ compile error
\item libnghttp2-\/dev version 0.\+6.\+4.\+2 -\/ to much nghttp2 Bugs ( required 1.\+22.\+0 or higher )
\item libssl-\/dev version 1.\+0.\+1 -\/ A\+L\+PN protocol\+: h2 is not negotiated error message ( required version 1.\+0.\+2 ) \section*{solution for libssl-\/dev and libnghttp2-\/dev}
\end{DoxyItemize}

add two \href{https://github.com/superjamie/lazyweb/wiki/Raspberry-Pi-Debian-Backports}{\tt repositorys} stretch and jessie-\/backports into /etc/apt/sources.list file. get libssl-\/dev and libnghttp2-\/dev 
\begin{DoxyCode}
apt-get -t jessie-backports install libssl-dev
apt-get -t stretch install libnghttp2-dev
\end{DoxyCode}
 after remove repositorys and run apt-\/get update. F\+I\+N\+I\+SH. 